\documentclass[11pt]{article}

\usepackage{natbib}
\usepackage{setspace}
\usepackage[left=2.5cm,top=2.8cm,right=2.5cm,bottom=2.8cm]{geometry}
\usepackage{graphicx}
\usepackage{amsmath}
\usepackage{theorem}
\usepackage{version}
\usepackage{pstricks}
\usepackage{multirow}

\usepackage{amssymb}

\setcounter{MaxMatrixCols}{10}
\onehalfspacing
\newtheorem{theorem}{Theorem}
\newtheorem{acknowledgement}{Acknowledgement}
\newtheorem{algorithm}{Algorithm}
\newtheorem{assumption}{Assumption}
\newtheorem{axiom}{Axiom}
\newtheorem{case}{Case}
\newtheorem{claim}{Claim}
\newtheorem{conclusion}{Conclusion}
\newtheorem{condition}{Condition}
\newtheorem{conjecture}{Conjecture}
\newtheorem{corollary}{Corollary}
\newtheorem{criterion}{Criterion}
\newtheorem{definition}{Definition}
\newtheorem{example}{Example}
\newtheorem{exercise}{Exercise}
\newtheorem{lemma}{Lemma}
\newtheorem{notation}{Notation}
\newtheorem{problem}{Problem}
\newtheorem{proposition}{Proposition}
{\theorembodyfont{\normalfont}
\newtheorem{remark}{Remark}
}
\newtheorem{summary}{Summary}
\newenvironment{proof}[1][Proof]{\textbf{#1.} }{\hfill \rule{0.5em}{0.5em} \bigskip}
\newenvironment{soln}[1][Soln]{\textbf{#1:} }{\hfill \rule{0.5em}{0.5em}}
\renewcommand{\cite}{\citeasnoun}
\renewcommand{\theenumii}{(\alph{enumii})}
\renewcommand{\labelenumii}{\theenumii}
\renewcommand{\theenumiii}{\roman{enumiii}}
\renewcommand{\labelenumiii}{\theenumiii.}
\begin{document}


\begin{center}
\textbf{Advanced Microeconomics S14.2\\}
\textbf{Ruochen Zhou}
\end{center}

\begin{enumerate}

\item Consider an exchange economy with two goods and two consumers, with
 	\begin{equation*}
	u^{1}(x_{1},x_{2})=x_{1}^{1/2}x_{2}^{1/2}\text{ and }u^{2}(x_{1},x_{2})=x_{1}^{1/2}+x_{2}^{1/2}.
	\end{equation*}
	Total endowments are $(10,10)$.
	\begin{itemize}
	\item[(a)] Suppose that a social planner wants to allocate goods to maximize consumer 1's utility while holding consumer 2's utility at $u^{2}(x_{1}^{2},x_{2}^{2})=4$. Find the assignment of goods to consumers that solves the planner's problem.
	\smallskip\\\\
	For consumer 2, we have\\
	$MRS_{12}^2=\dfrac{(x_2^2)^{1/2}}{(x_1^2)^{1/2}}$\\
	$u^2(x_1,x_2)=(x_1^2)^{1/2}+(x_2^2)^{1/2}=4$\\\\
	For consumer 1, we have\\
	$MRS_{12}^1=\dfrac{x_2^1}{x_1^1}$\\\\
	At equilibrium, we must have\\
	$MRS_{12}^1=MRS_{12}^2\Rightarrow\dfrac{x_2^1}{x_1^1}=\dfrac{(x_2^2)^{1/2}}{(x_1^2)^{1/2}}$\\
	Since $e_1=e_2$, we must have\\
	$\Rightarrow x_1^{2*}=x_2^{2*}\Rightarrow\textbf{x}^{2*}=(4,4)$\\\\
	Our market clearing conditions must hold\\
	$x_1^{1*}+x_1^{2*}=e_1=10$\\
	$x_2^{1*}+x_2^{2*}=e_2=10$\\
	$\Rightarrow\textbf{x}^{1*}=(6,6)$\\
	\item[(b)] Suppose that the planner just divides the endowments so that $\mathbf{e}^{1}=(10,2)$ and $\mathbf{e}^{2}=(0,8)$ and then lets the consumers transact through perfectly competitive markets. Find the Walrasian equilibrium and show that the WEA is the same as the solution in part (a).
	\smallskip\\\\
	For consumer 1, we have\\
	$y^1=10p_1+2p_2$\\
	$MRS_{12}^1=\dfrac{x_2^1}{x_1^1}=\dfrac{p_1}{p_2}$
	$\Rightarrow x_1^{1*}=\dfrac{10p_1+2p_2}{2p_1}=5+\dfrac{p_2}{p_1}\Rightarrow x_2^{1*}=\dfrac{10p_1+2p_2}{2p_2}=5\dfrac{p_1}{p_2}+1$\\\\
	For consumer 2, we have\\
	$y^2=8p_2$\\
	$MRS_{12}^2=\dfrac{(x_2^2)^{1/2}}{(x_1^2)^{1/2}}=\dfrac{p1}{p2}\Rightarrow x_1^{2*}=\dfrac{8p_2^2}{p_1(p_1+p_2)}\Rightarrow x_2^{2*}=\dfrac{8p_1}{p_1+p_2}$\\\\
	Our market clearing condition must hold\\
    $x_2^{1*}+x_2^{2*}=e_2^1+e_2^2=2+8=10\Rightarrow5\dfrac{p_1}{p_2}+1+\dfrac{8p_1}{p_1+p_2}=10$\\
	$\Rightarrow 5\dfrac{p_1^2}{p_2}+5p_1+8p_1=9p_1+9p_2$\\
	$\Rightarrow 5(\dfrac{p_1}{p_2})^2+4\dfrac{p1}{p_2}-9=0$\\
	We can solve for $\dfrac{p_1}{p_2}$ using quadratic formula\\
	$\dfrac{p_1}{p_2}=\dfrac{-4+\sqrt{4^2-4(5)(-9)}}{2(5)}=\dfrac{-4+14}{10}=1\Rightarrow\textbf{x}^{1*}=(6,6)\Rightarrow\textbf{x}^{2*}=(4,4)$
	\end{itemize}
\pagebreak
\item Consider an exchange economy with two goods and two consumers, with
 	\begin{equation*}
	u^{1}(x_{1},x_{2})=x_{1}^{1/2}x_{2}^{1/2}\text{ and }u^{2}(x_{1},x_{2})=\min\{x_{1},x_{2}\}.
	\end{equation*}
	Total endowments are $(12,10)$.
	\begin{itemize}
	\item[(a)] Suppose that a social planner wants to allocate goods to maximize consumer 1's utility while holding consumer 2's utility at $u^{2}(x_{1}^{2},x_{2}^{2})=4$. Find the assignment of goods to consumers that solves the planner's problem.
	\smallskip\\\\
	For consumer 2, we have\\
	$x_1^2=x_2^2$\\
	$u^2(x_1,x_2)=min\{x_1^2,x_2^2\}=4\Rightarrow\textbf{x}^{2*}=(4,4)$\\\\
	Our market clearing conditions must hold\\
	$x_1^{1*}+x_1^{2*}=e_1=12$\\
	$x_2^{1*}+x_2^{2*}=e_2=10$\\
	$\Rightarrow\textbf{x}^{1*}=(8,6)$\\
	\item[(b)] Suppose that the planner just divides the endowments so that $\mathbf{e}^{1}=(4,9)$ and $\mathbf{e}^{2}=(8,1)$ and then lets the consumers transact through perfectly competitive markets. Find the Walrasian equilibrium and show that the WEA is the same as the solution in part (a).
	\smallskip\\\\
	For consumer 1, we have\\
	$y^1=4p_1+9p_2$\\
	$MRS_{12}^1=\dfrac{x_2^1}{x_1^1}=\dfrac{p_1}{p_2}$
	$\Rightarrow x_1^{1*}=\dfrac{4p_1+9p_2}{2p_1}=2+\dfrac{9}{2}\dfrac{p_2}{p_1}\Rightarrow x_2^{1*}=\dfrac{4p_1+9p_2}{2p_2}=2\dfrac{p_1}{p_2}+\dfrac{9}{2}$\\\\
	For consumer 2, we have\\
	$y^2=8p_1+p_2$
	$x_1^2=x_2^2\Rightarrow x_1^{2*}=\dfrac{8p_1+p_2}{p_1+p_2}\Rightarrow x_2^{2*}=\dfrac{8p_1+p_2}{p_1+p_2}$\\\\
	Our market clearing condition must hold\\
    $x_2^{1*}+x_2^{2*}=e_2^1+e_2^2=9+1=10\Rightarrow2\dfrac{p_1}{p_2}+\dfrac{9}{2}+\dfrac{8p_1+p_2}{p_1+p_2}=10$\\
    $\Rightarrow 4\dfrac{p_1^2}{p^2}+4p_1+16p_1+2p_2=11p_1+11p_2$\\
    $\Rightarrow 4(\dfrac{p_1}{p^2})^2+9\dfrac{p_1}{p_2}-9=0$\\
    We can solve for $\dfrac{p_1}{p_2}$ using quadratic formula\\
	$\dfrac{p_1}{p_2}=\dfrac{-9+\sqrt{9^2-4(4)(-9)}}{2(4)}=\dfrac{-9+15}{8}=\dfrac{3}{4}\Rightarrow\textbf{x}^{1*}=(8,6)\Rightarrow\textbf{x}^{2*}=(4,4)$
	\end{itemize}
\end{enumerate}

\end{document}

